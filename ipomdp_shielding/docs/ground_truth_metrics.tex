\documentclass[11pt]{article}
\usepackage{amsmath, amssymb}
\usepackage[margin=1in]{geometry}

\title{Ground Truth Comparison Metrics}
\author{}
\date{}

\begin{document}
\maketitle

\section{Setup}

Given $R$ Monte Carlo runs over $T$ steps, let $\mathcal{A}$ denote the action set. For each run $r$ at step $t$:
\begin{itemize}
    \item $s_t^{(r)}$ is the true state.
    \item $\mathcal{P}_t^{(r)}$ is the LFP belief polytope after propagation.
    \item $S_{\text{safe}}(a)$ is the set of state indices where action $a$ is safe (from the inverted shield).
\end{itemize}

Define the \textbf{predicted minimum safety probability} for action $a$ in run $r$ at step $t$:
\[
    \hat{p}_{t,a}^{(r)} \;=\; \min_{b \in \mathcal{P}_t^{(r)}} \sum_{i \in S_{\text{safe}}(a)} b_i
\]

Define the \textbf{empirical safety indicator} for action $a$ in run $r$ at step $t$:
\[
    \mathbb{1}_{t,a}^{(r)} \;=\;
    \begin{cases}
        1 & \text{if } s_t^{(r)} \in S_{\text{safe}}(a) \\
        0 & \text{otherwise}
    \end{cases}
\]

Let $R_t \leq R$ be the number of runs that have not terminated by step $t$. The step-averaged quantities are:
\[
    \bar{p}_{t,a} \;=\; \frac{1}{R_t} \sum_{r=1}^{R_t} \hat{p}_{t,a}^{(r)}, \qquad
    \bar{e}_{t,a} \;=\; \frac{1}{R_t} \sum_{r=1}^{R_t} \mathbb{1}_{t,a}^{(r)}
\]

\section{Action Coverage Rate}

The action coverage rate measures the fraction of (step, action) pairs where the empirical safety frequency meets or exceeds the predicted minimum:
\[
    \texttt{action\_coverage\_rate} \;=\; \frac{\bigl|\{(t,a) : \bar{e}_{t,a} \geq \bar{p}_{t,a}\}\bigr|}{T \cdot |\mathcal{A}|}
\]

A value close to 1 indicates that the LFP lower bounds are sound in aggregate: the predicted minimum rarely overestimates the true safety frequency.

\section{Mean Conservatism Gap}

The mean conservatism gap quantifies how conservative the predicted lower bounds are on average:
\[
    \texttt{mean\_conservatism\_gap} \;=\; \frac{1}{T \cdot |\mathcal{A}|} \sum_{t=1}^{T} \sum_{a \in \mathcal{A}} \bigl(\bar{e}_{t,a} - \bar{p}_{t,a}\bigr)
\]

A positive value indicates the predictions are conservative (the true frequency exceeds the predicted minimum). A value near zero indicates tight bounds.

\end{document}
